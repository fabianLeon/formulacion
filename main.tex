\documentclass{report}
\usepackage[utf8]{inputenc}
\usepackage[spanish]{babel}
\usepackage{makeidx}

\title{ESTÚDIO DE MERCADO CANECA INTELIGENTE}
\author{Fabián Sánchez, fadarsaleeing@gmail.com \and Fabián García, frgarciap@correo.udistrital.edu.co \and Kevin Ibarra, kevini0501@gmail.com \and Samuel Diaz, samueldiaz@gmail.com}

\date{Marzo 08 de 2017}

\begin{document}

\maketitle


 \tableofcontents



\chapter{Identificación del Bien}

    \section{Usos}
    Corresponde a un grupo de 3 canecas en las cuales se reparten los materiales en; no reciclables, plásticos y papel, que permiten separar los residuos sólidos en un hogar o empresa pequeña o mediana de cualquier naturaleza, en el cual el volumen de los residuos se pueda almacenar durante un corto lapso mientras las autoridades competentes se encargan de la recolección y manejo de estos materiales.

    \section{Usuarios}
    Como se mencionó anteriormente son las familias bogotanas de estratos 3,4,5 y 6 las cuales por sus condiciones socio-económicas están en capacidad de adquirir un dispositivo que no es de interés prioritario en la canasta familiar, pero facilita las labores de reciclaje en el hogar, que son poco comunes en la ciudad. Adicionalmente las PYMES (pequeñas y medianas empresas), también pueden ser usuarios finales, dado que las dimensiones del producto no son aptas para grandes volúmenes de residuos sólidos, que se podrían presentar en una empresa de tamaño considerable.

    \section{Presentación}
    Son un grupo de 3 canecas cada una de 53 litros para interiores, con tapa, el fondo de la caneca es de 34 cm, altura 67 cm y 24 cm de ancho, una de color verde para residuos orgánicos (no reciclables), una de color gris para los papeles y otra de color azul para plásticos, ensamblados en una estructura de metal liviano que permita la ubicación cómoda del producto en un espacio determinado. En la zona superior se implementará una interfaz electrónica que tendrá el usuario para facilitar el proceso de selección que se realizará de manera automática.

    \section{Producto}
    Inicialmente el proyecto se enfoca en un único producto que corresponde a un sistema de selección de residuos con los tamaños indicados anteriormente, no producirá ningún tipo de desechos dado sus características de uso. Según la descripción dada el tiempo de vida está determinado principalmente por las características del sistema de automatización ya que tanto los recipientes de plástico como la estructura de metal tienen un tiempo de vida largo (5-10 años), según los cuidados y mantenimiento que se le den al producto, por ello es necesario en el estudio técnico definir los materiales electrónicos y mecánicos bajo esta premisa; hacer duradero el producto ya que no será un bien de rotación masiva que las personas estén comprando semanal, ni mensualmente. Debido a que no es un producto que este ofrecido en el mercado o que las personas del común conozcan es necesarios hacer campañas de instrucción en el uso del sistema, para evitar daños constantes que den una mala imagen al vendedor del producto y necesidad de constante mantenimiento sobre el sistema que implicaría aumentos en el precio del producto, sobre la instrucción al usurario se mencionará en detalle en los procesos de comercialización. Otra de las características del bien, al ser algo novedoso en el mercado es que no es sencillo reconocer una época del año en el cual sea la temporada alta de ventas.

    \section{Sustitutos}
    Sin lugar a dudas este bien presenta productos sustitutos las típicas canecas usadas en los hogares y las empresas, ya que son más económicas, debido a que no corresponden a un bien que incluye tecnología.

    \section{Complementarios}
    El sistema para su funcionamiento requiere únicamente de una conexión a la red eléctrica, sin embargo, para mejorar la experiencia con el producto se recomienda no usar los recipientes directamente como los contenedores de los residuos, sino un conjunto de bolsas, preferiblemente de colores que ayuden a las entidades públicas a continuar con el proceso de selección sin perder los esfuerzos realizados directamente en la fuente de generación de residuos (hogares y empresas).

    El bien presentado corresponde a un bien de consumo final, pues no hace parte de ningún proceso industrial, sino que son directamente los usuarios quieres obtienen el beneficio a la hora de seleccionar los residuos sólidos que produzcan.

    \section{Insumos}

    Como el proyecto se basa en la idea de darle un valor agregado a las típicas canecas ya existentes, el mercado de insumos, está bien establecido incluso con precios de venta al usuario final que son conocidos fácilmente en las tiendas online de las empresas que fabrican por ejemplo las canecas que se quieren intervenir. De esta manera se evidencia que hay un mercado donde existen un grupo variado de empresas que ofrecen productos similares con diferencias muy particulares de acuerdo a la aplicación, algo que será determinado en el estudio técnico.

    Al ser una iniciativa que busca mejorar el proceso de separación de residuos sólidos, se puede llegar a tener apoyo del gobierno distrital o nacional, sin embargo, esto no se planteará en la etapa inicial del proyecto debido a las limitaciones temporales del mismo.

    \section{Precio}
Como se mencionó anteriormente este producto no se encuentra ofrecido en el mercado nacional, de modo que para determinar rangos de precios es necesario buscar en enlaces ubicados en el extranjero que de todas maneras aún no comercializan el producto dentro del país de modo que de entrada no se cuenta con una referencia de precio para entrar al mercado

    \section{DOFA}

\chapter{La demanda}

    \section{Variables}
    La demanda del producto puede depender de variables como, el precio al que se ofrezca el producto, el precio al que se ofrezcan los bienes sustitos de este como lo son las canecas convencionales, el precio al que se ofrezcan los bienes complementarios como los son las bolsas de almacenamiento y la energía eléctrica necesaria,  las regulaciones gubernamentales que puedan llegar a generarse debido a los niveles de contaminación material y el agotamiento de espacio en el relleno sanitario de la ciudad,  la postura que adopte la sociedad con respecto a las dificultades que enfrenta y enfrentara el medio ambiente en un futuro, la capacidad de la gente para organizar y almacenar desechos por ellos mismos y la sensación de durabilidad y fiabilidad que el usuario tenga sobre el producto.

    \section{Área de mercado}
    El producto será ofrecido a todo el mercado sin importar condiciones económicas o sociales, pero se espera que los demandantes del producto sean  familias de estrato 3, 4, 5 y 6 teniendo en cuenta parámetros ya nombrados. Teniendo en cuenta esto, la localización espacial de la zona geográfica a la que está dirigida el producto serian barrios como los de la localidad de Usaquén , Chapinero, ciertas zonas de Suba y Teusaquillo; donde según la secretaria de planeación, en el documento del 2014 referente a la cantidad de hogares que hay por estrato en cada barrio, Usaquén cuenta con alrededor de 152414 hogares de estratos entre 3 y 6, Chapinero con 48188, mientras que Suba y Teusaquillo tiene una gran densidad de hogares entre el estrato 3 y 4 con mas modestos números entre el 5 y 6.

    Así mismo, en la encuesta multipropósito del 2014 publicada por la secretaria de planeación, se muestra que  alrededor del 36\% de la población de la ciudad es de estrato 3 con un crecimiento exponencial con respecto al año 2011 de 1.79, el 7.8\%  es de estrato 4 con un crecimiento exponencial de -3.3, el 2.6\% es estrato 5 con un crecimiento exponencial de 2.3 5 y el 1.9\% es de estrato 6 con un crecimiento exponencial de 2.66. De modo que el mercado al que está dirigido nuestro producto tiende a aumentar con respecto al tiempo.

    También se espera que organizaciones estatales y privadas y cierto tipo de almacenes sean factibles compradores del producto, puesto que está visto que centros comerciales, empresas, colegios, universidades y otro tipo de establecimientos han implementado dispositivos para la organización de desechos sólidos.

    \section{Comportamiento histórico}
    Ya que esta es una idea innovadora que ofrece un concepto nuevo para la sociedad, no hay antecedentes sobre la aceptación de este producto en el mercado. El único parámetro histórico que se podría tener en cuenta es la aceptación que han tenido en las grandes organizaciones las canecas separadoras de desechos que no cuentan con el valor agregado con el que sí cuenta nuestro producto.

    \section{Segmentación de mercado}

    \subsection{Potencial número de compradores}
    Se espera que el número de compradores sea proporcional al número de hogares en los estratos 3, 4, 5 y 6; es necesario hacer un sondeo a los posibles clientes, pero para esto asumimos necesario hacer primero un estudio técnico para determinar un precio tentativo del producto y así poder ver la aceptación del producto en la sociedad.

    \subsection{Lugares de venta}
    Las ventas se efectuaran en locales específicos ubicados estratégicamente en las localidades y barrios de mayor capacidad monetaria, ya que se espera que estos sean los compradores mayoritarios del producto. Igualmente la venta se hará también vía internet, ya que se cuenta con la experiencia de un ingeniero de sistemas, mediante las redes sociales que igualmente difundirán una página de internet con información apropiada sobre el producto

    \subsection{Motivación de compra}
    El comprador se verá motivado a adquirir el producto teniendo en cuenta el valor agregado de este, la automatización del proceso de separación de residuos.

    \subsection{Características de los compradores}
    Se aspira a que los compradores sean miembros activos de una familia, de edades adultas, interesados por el bienestar del medio ambiente, con un nivel de ingresos robusto, puesto que quizá la población con ingresos más modestos no considere necesario el producto y por consiguiente no esté dispuesta a comprarlo. Parámetros como la edad, el sexo o la raza del comprador son totalmente irrelevantes.

    \section{Consumo aparente}
    El consumo aparente tiene en cuenta la producción estimada, las importaciones, exportaciones y la variación de los inventarios, pero como el producto está dirigido a corto plazo a un zona especifica de Bogota D.C, y las importaciones no son necesarias, el concepto de consumo aparente no se trabajara

    \section{Estimación de la demanda futura}
    Posterior a la producción inicial, al hacer los pertinentes estudios sobre la aceptación del producto, se empezara hacer una estimación apropiada sobre la demanda en tiempos posteriores; esto utilizando diferentes métodos.

\chapter{La oferta}

\chapter{Precio}

    \section{Competencia}
    El negocio de las canecas de basura, es un negocio muy movido, ya que pues, todas las personas necesitan de al menos un cesto, para arrojar los deshechos en sus hogares. Por esta razón se pueden encontrar cestos de muchos tamaños, formas, colores y de igual manera una gran variedad de precios, que varían dependiendo de las características de cada cesto.
Recorriendo almacenes de cadena en la ciudad de Bogotá, lo que se encontró es que, una caneca plástica sencilla, de color y no muy grande, costaba alrededor de \$40.000. Sin embargo, se sabe que el fuerte de los almacenes de cadena no es, precisamente la comercialización de las canecas.
 Por lo que se siguió con la investigación, indagando en las diferentes páginas comerciales disponibles, como mercado libre u olx y también, en algunas tiendas especializadas en la comercialización de productos de aseo. Los resultados fueron interesantes, ya que se pudo encontrar una mayor variedad de cestas y, a un muy buen precio.

    \section{Tipos de cestos encontrados}
        \begin{itemize}
        \item Plástico-pequeñas: Entre las descripciones decía, que su uso más común es para albergar los residuos de las habitaciones; las hay de muchos colores, los más comunes eran el rojo, verde, azul, gris y negro. Tenían un costo de \$28.000 aproximadamente.

        \item Plástico-medianas: Con casi las mismas características de las pequeñas, salo el tamaño. Según las descripciones, estas canastas se utilizan comúnmente para los residuos del baño y, en algunas ocasiones los residuos que surgen en la sala y su costo estaba entre los \$38.000 a \$72.000 (Que fue la más costosa que se encontró)

        \item Plástico-Grandes: Las características generales (Color y forma), son las mismas que en los dos casos anteriores, dentro de las características mencionadas estaba el uso para “patios”, ya que según los vendedores se usan mucho en colegios y empresas, porque gracias a su gran tamaño, pueden albergar los residuos de las muchas personas que hay en la instalación, También son utilizadas para albergar residuos de la cocina. El costo de estas canecas estaba entre los \$80.000 y \$126.000.

        \item Aluminio: Respecto a la forma, solo se encontraron dos modelos, cilíndricas y cuadradas. En cuanto al color solamente se encontraron en diferentes tonos de gris brillante. Ahora bien, una pequeña costaba entre los \$112.000 y \$147.000 y una grande costaba entre \$312.000 y \$380.000.


        \end{itemize}

    Aunque como se puede ver, hay una gran variedad de canecas, no fue posible encontrar una caneca inteligente. Lo más “inteligente”, que se podía encontrar era una caneca doble, o, canecas triples que tenían el nombre de qué tipo de residuo arrojar en ellas. Esto es un factor muy positivo, ya que aunque la competencia de canecas es muy fuerte y amplia, no hay competencia de canecas inteligentes o, automáticas en el mercado.

    \section{Materiales}
    Un factor muy importante, a la hora de escoger el precio de un producto o servicio es, garantizar alguna ganancia. Es decir, que el producto o servicio valga más, de lo que cueste hacerlo. En el caso de la cesta automática de basura, es de vital importancia analizar el costo de los materiales y mano de obra necesarios para elaborar cada unidad.

    Haciendo una aproximación de los materiales, como la carcasa plástica, el cable necesario para las conexiones, los micros, resistencias y sensores necesarios para la construcción del sistema inteligente, la baquela en donde se va a ubicar el circuito, unos LED que indique cuando está llena la cesta, se estiman unos gastos aproximadamente de entre \$150.000 a \$160.000. Esto suponiendo que los kits de ensamble y programación ya estén disponibles.

    En cuanto a la mano de obra, hay que tener en cuenta que hay dos etapas, la programación y el ensamble. El ingeniero Fabián, será el encargado de la programación del sistema de la cesta y del ensamble, se encargará el ingeniero René. Se tiene presupuestado un pago de alrededor de \$130.000, que será dividido, en un 60\% para el ensamblador y 40\% para el programador, ya que el ensamble consume una mayor cantidad de tiempo que la programación del sistema.

    \section{Elección del precio}
    Para elegir el precio del artefacto, se deben tener en cuenta los puntos anteriormente tocados, como el de competencia y materiales y mano de obra. Por lo tanto, la cesta debe costar entre \$28.000 y \$400.000 y además, debe valer más de lo que suman el costo de materiales y el de mano de obra, además hay que incluir gastos de publicidad y distribución, por lo que el precio neto de cada cesta deberá ser de \$380.000.

\chapter{Comercialización}
    Lo primero a tener en cuenta para llevar a cabo una buena comercialización de producto es, escoger qué canal de comercialización se va a escoger. Para ello es necesario responder algunos interrogantes, que ayudarán a dar con el mejor tipo de canal:

    \section{¿Cuál es el grado de concentración geográfica del mercado?}
    El mercado de las canecas es muy amplio, por lo que no podemos hablar de concentración de este, y es que no solamente se pueden comprar dentro de Bogotá, sino que también se pueden comprar en otra ciudad del país, a muy buen precio y pagar el envío, que no es muy costoso. Sin embargo, esto es, hablando de las canecas comunes, pero como ya se había mencionado antes no hay competencia directa para una caneca inteligente, por lo que sí se puede lograr un buen precio para las cestas inteligentes, se deberán realizar estudios para ver en qué parte de la ciudad tendría el mejor impacto de aceptación y económico, para lanzar una prueba piloto.

    \section{¿Qué tipo de distribuidores existen?}
    Distribuidores, los hay de todo tipo (mayoristas, minoristas y agentes), sin embargo los que más se ven, son los distribuidores de tipo mayoristas, ya que como se ha mencionado antes, es fácil encontrar canecas de distinto tamaño, color, forma, material.

    \section{¿Cuáles son los mecanismos usados en las ventas?}
    A menudo, no se ve mucha propaganda de cestas de basura en internet, o, en la televisión, ya que casi en todos los hogares es indispensable, al menos una cesta, para arrojar los residuos, entonces las empresas que producen cestas, lo hacen por medio de lotes, que contienen una gran cantidad de cestas y las reparten al por mayor a los vendedores.

    \section{¿Cuáles son los compradores principales y dónde están localizados?}
    Hablar de compradores principales es complicado, ya que, en todos los hogares hay al menos una cesta para basura y es que es algo necesario, por higiene y estética, por lo que si se va a hablar de localización tendrá que hablarse de toda Bogotá. Sin embargo, ya que el producto está dirigido más para hogares, que para corporaciones y empresas, hay que esperar los resultados que arrojen los estudios y encuestas para saber en qué sector de la comunidad se va a entrar.

    \section{¿Cuáles son las normas vigentes con respecto a los comercializadores?}

    Con base en las respuestas a dichos interrogantes, entre los integrantes del proyecto, se decidió que el canal más viable era entablar una relación directa entre, vendedor y comprador, ya que como es un producto nuevo, es necesario convencer al comprador de porqué, debe adquirir este instrumento. El encargado de la parte de negociaciones con el cliente es el ingeniero Kevin Ibarra.

    \section{Transporte y Publicidad}
    Luego de tener claro el canal de comercialización que se va a utilizar, es importante tener en cuenta dos aspectos importantes para una buena comercialización del artefacto, como lo son el transporte (Que incluye el almacenamiento) y la publicidad.

    Para el transporte, una vez teniendo el vehículo para poder realizar los envíos, se podría decir que los gastos más representativos que se pueden presentar, son la gasolina y el parqueadero. El envío de cada unidad tendrá un costo de aproximadamente \$10.000, a cualquier parte de la ciudad. Con esos \$10.000, se pueden cubrir gastos de parqueadero, transporte y una pequeña reserva para posibles accidentes inesperados.

    En el caso de la publicidad, como ya se mencionó anteriormente, no es común encontrar anuncios promocionando algún tipo de cesta, por lo que digamos que, comercialmente no es tan importante una gran inversión en publicidad. Sin embargo es una labor que tiene que hacerse y, como la idea es aliarse con las autoridades ambientales competentes, más que comerciales  y anuncios, es más importante las impresiones de folletos y pancartas, para regalar a los asistentes a las conferencias, que se hagan en conjunto con las autoridades, de esta manera se puede obtener un buen alcance publicitario.

    En cuanto a los costos de los folletos, el ingeniero Santiago se encargará del diseño de los mismos y las impresiones costarán alrededor de \$550.

    \section{Margen de Ganancia}
    Luego de haber analizado factores del instrumento como, el precio y la comercialización, pasaremos a analizar el margen de ganancia, para ver si son viables los valores elegidos en el documento.

    \[Mg = P_{c} - P_{v}\]

    Donde, Mg es el margen de ganancia, Pc el precio al que se vende y Pv lo que nos cuesta hacer una unidad. Reemplazando los valores netos:

    \[Mg = \$ 380.000 - \$ 290.000\]
    \[Mg = \$ 90.000\]

    Como se pudo ver, el margen de ganancia arroja un valor positivo, y es de \$90.000 que representa una utilidad del 23.6\%, que es un valor muy bueno.

    \section{conclusiones}

\end{document}
Contact GitHub API Training Shop Blog About
© 2017 GitHub, Inc. Terms Privacy Security Status Help
